% Otto-von-Guericke-Universität Magdeburg
% Fakultät für Mathematik
%
% Vorschlag zur Gestaltung von Abschlussarbeiten mit LaTeX
%
%
%\documentclass[a4paper, 12pt]{scrreprt} % KOMA-Script-Report
%
\documentclass[a4paper, 12pt]{scrartcl} % KOMA-Script-Article 
%                                      (dann keine \chapter möglich!!!)
%                                      (erfodert Anpassung in FMA-commands.tex) 
%----LaTeX-Pakete-----------------
\usepackage[ngerman,american]{babel}   %Ueberschriften in Englisch
%\usepackage[american,ngerman]{babel}  %Ueberschriften in Deutsch
\usepackage[utf8x]{inputenc} % Quelltext mit Umlauteingabe

\usepackage{geometry}
\geometry{twoside, outer=25mm, inner=30mm, top=30mm, bottom=30mm} % zweiseitig
% \geometry{outer=25mm, inner=30mm, top=30mm, bottom=30mm}        % einseitig
\usepackage{latexsym}          % spezielle Mathematik-Symbole
\usepackage{graphicx}          % zum Einbinden von Bildern
\usepackage{amsmath, amssymb}  % LaTeX-Erweiterungen der AMS

% ----Code Syntax und Highlighting 
%\usepackage[T1]{fontenc}
%\usepackage{beramono}
%%\usepackage{listings}
%\usepackage[usenames,dvipsnames]{xcolor}

%%
%% Julia definition (c) 2014 Jubobs
%%
%\lstdefinelanguage{Julia}%
%  {morekeywords={abstract,break,case,catch,const,continue,do,else,elseif,%
%      end,export,false,for,function,immutable,import,importall,if,in,%
%      macro,module,otherwise,quote,return,switch,true,try,type,typealias,%
%      using,while},%
%   sensitive=true,%
%   alsoother={$},%
%   morecomment=[l]\#,%
%   morecomment=[n]{\#=}{=\#},%
%   morestring=[s]{"}{"},%
%   morestring=[m]{'}{'},%
%}[keywords,comments,strings]%
%
%\protected\def\dg{\ensuremath{^\circ}}
%\DeclareUnicodeCharacter{05A8}{\dg}
%
%
%\lstset{%
%    language         = Julia,
%    basicstyle       = \ttfamily,
%    keywordstyle     = \bfseries\color{blue},
%    stringstyle      = \color{magenta},
%    commentstyle     = \color{ForestGreen},
%    showstringspaces = false,
%    inputencoding=utf8x
%}
%
%\lstdefinestyle{CStyle}{
%    backgroundcolor=\color{backgroundColour},   
%    commentstyle=\color{mGreen},
%    keywordstyle=\color{magenta},
%    numberstyle=\tiny\color{mGray},
%    stringstyle=\color{mPurple},
%    basicstyle=\footnotesize,
%    breakatwhitespace=false,         
%    breaklines=true,                 
%    captionpos=b,                    
%    keepspaces=true,                 
%    numbers=left,                    
%    numbersep=5pt,                  
%    showspaces=false,                
%    showstringspaces=false,
%    showtabs=false,                  
%    tabsize=2,
%    language=C
%}
%

% ----Julia Code visibility-----
\usepackage[theme = default-plain]{jlcode}

\captionsetup[lstlisting]{format=listing,labelfont=white,textfont=white,margin=0pt}


% ----Einbinden von pdf-Dateien oder einzelnen Seiten daraus
\usepackage{pdfpages}
\usepackage{xspace}
\usepackage{setspace}        % zum Variieren des Zeilenabstandes
% Anwendung im Text:  \includepdf[pages={7, 9-12}]{Name der pdf-Datei}
%------
% Kopf- und Fußzeilen mit scrpage2 ---------------*Beginn*------------
\usepackage[automark,headsepline]{scrpage2}
\automark[section]{chapter}
\pagestyle{scrheadings}			
\clearscrheadfoot
\ihead[]{\headmark}
\cfoot[\hfill -- \thepage{} -- \hfill]{\hfill -- \thepage{} -- \hfill}
\setkomafont{pagefoot}{\normalfont\sffamily}
% Kopf- und Fußzeilen mit scrpage2 ---------------*Ende* --------------
\setlength{\parindent}{0cm}
\setcounter{tocdepth}{4}       % 4 Gliederungsebenen in das Inhaltsverzeichnis
\setcounter{secnumdepth}{3}    % 4 (!) Gliederungsebenen werden nummeriert(0-3) 
\input{FMA-commands}           % hilfreiche LaTeX-Befehlsabkürzungen
% ------------------------------------------------------------------------------
%
%###############################################################################
\begin{document}
\pagenumbering{Roman}
% Deckblatt der Abschlussarbeit
% nichts von der Formatierung verändern
%
% Einträge nur unterhalb der Zeile 
%####### hier aktualisieren
% bearbeiten
%
%

\begin{titlepage}
  \begin{center}
    \vspace*{1,5cm}
    \begin{Large}
      \doublespacing
      \begin{scshape}
        % ####### hier aktualisieren
        Julia : A Fresh Approach to Numerical Computation
      \end{scshape}
      \singlespacing
    \end{Large}
    \vspace{\fill}
    Faculty of Mathematics\\
    Otto-von-Guericke-Universität Magdeburg\\
    %zur Erlangung des akademischen Grades \\
    % Diplom-Mathematiker Diplom-Wirtschaftsmathematiker
    % Diplom-Technomathematiker Diplom-Computermathematiker
    %Bachelor of Science
    % Master of Science
    %\\
    %angefertigte\\
    \vspace{\fill}
    \begin{Large}
      % ####### hier aktualisieren durch Umsetzen des Kommentarzeichens %
      \textsc{Scientific Computation Seminar 2021}
      % \textsc{Bachelorarbeit}
      % \textbf{Diplomarbeit} \textbf{Masterarbeit}
      \\
    \end{Large}
    \vspace{\fill}

    by\\
    \textsc{Carl Julius Martensen}\\
    born on 31.10.1986, Preetz,\\
    %Studiengang Mathematik,\\
    %Studienrichtung Wirtschaftsmathematik.\\[2ex]
    \today\\
    \vspace{\fill} Supervised at the Max-Planck-Institut für Dynamik komplexer
    technischer Systeme
    % Algebra und Geometrie
    % Analysis und Numerik
    % Mathematische Optimierung Mathematische Stochastik
    by\\
    \begin{scshape}
      Dr. Jens Saak
    \end{scshape}
  \end{center}
\end{titlepage}

%%% Local Variables: 
%%% mode: latex
%%% TeX-master: "FMA-Vorlage"
%%% TeX-PDF-mode:t
%%% auto-fill-function:nil
%%% mode:auto-fill
%%% flyspell-mode:nil
%%% mode:flyspell
%%% ispell-local-dictionary: "american"
%%% End: 


\tableofcontents
\clearpage
% \addcontentsline{toc}{section}{Abbildungsverzeichnis}
% \listoffigures
% \cleardoublepage
\pagenumbering{arabic}
\setcounter{page}{1}      %Beginn der Textseitenzaehlung
% ------------------------------------------------------------------------------
% hier beginnen die einzelnen Kapitel der Arbeit
% The following text is only demonstrating the use of sections for structuring
% the contents. The section names are by no means obligatory.
\section{Introduction}
\label{sec:introduction}

\subsection{Basic Defintions and Notation}
\label{sec:basic-defint-notat}

\section{Overview}
\label{sec:overview}

\section{Experiments}
\label{sec:experiments}


%%% Local Variables: 
%%% mode: latex
%%% TeX-master: "FMA-Vorlage"
%%% TeX-PDF-mode:t
%%% auto-fill-function:nil
%%% mode:auto-fill
%%% flyspell-mode:nil
%%% mode:flyspell
%%% ispell-local-dictionary: "american"
%%% End: 

% 
\newpage

\section*{Erklärung}		
\thispagestyle{empty}				

Hiermit erkläre ich, dass ich die vorliegende Arbeit selbstständig und ohne
Benutzung anderer als der angegebenen Quellen und Hilfsmittel angefertigt habe.
\newline\newline 
Ort, Datum, Unterschrift
\end{document}
% for emacs user
%%% Local Variables: 
%%% mode:latex
%%% TeX-master:t
%%% TeX-PDF-mode:t
%%% auto-fill-function:nil
%%% mode:auto-fill
%%% flyspell-mode:nil
%%% mode:flyspell
%%% ispell-local-dictionary: "american"
%%% End: 
