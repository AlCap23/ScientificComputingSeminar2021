\section{Introduction}
\label{JM:sec:introduction}

In the world of today's sciences, the use of computers has transformed the field and requirements for researchers tremendously. 
The ability to collect and process data at high rates using (semi-) automatic approaches pushes the classical research fields like
physics, biology, medicine, engineering, and social sciences towards hybrid models and big data evaluation. A recent example is the detection of gravitational waves, 
where multiple software components have been used to process the data and help the researchers to classify the events \cite{JMMacleodEtAl2021} .\\


The requirements on software development used for research purposes have decreased below expert knowledge. Research software and coding are much more of a general tool, similar to text editors, for many scientific disciplines. 
As a result scientists from fields where programming and numerical computation play at most a minor role, e.g. psychology, sociology but also biology, start to develop code on their own.
This is reflected in the rise of high-level languages like \textit{Python}, \textit{Ruby}, \textit{R}, or \textit{Lua} \cite{JMVanRossumDrake2009,JMThomasEtAl2005, JMRCT2016, JMIerusalimschy2006} which are semantically easier to program as opposed to lower-level languages such as 
\textit{C}, \textit{C++}, \textit{Fortran}, \textit{Go}, and \textit{Rust} \cite{JMKernighanRitchie2006, JMStroustrup2013, JMBackusHeising1964, JMMeyerson2014, JMMatsakisKlockII2014}. 
The broader, more diverse user base of high level languages results in a more versatile ecosystem, giving rise to various packages and modules designed
for specific tasks with a clean interface. Despise these advantages, the price to be paid is commonly linked to the performance in terms of execution time and memory consumption of the resulting code. 
A common pattern used is hence to write an application programming interface (API) in high-level languages and move the computationally expensive parts to a low-level implementation after finishing prototyping. 
This is referred to as the \textit{two language problem}.
This approach, while being general, has some major drawbacks related to datatype availability, effort of maintanance, and dependency management - to name just a few.\\

To resolve this - and other - problems of scientific computation the \textit{Julia} programming language has been created \cite{JMBezansonEtAl2015}. It aims to provide performant code, leveraging the best features of 
prominent languages without complicated semantic structures, contradicting the common misconception that computational performance comes at the price of ease of use \cite{JMEdelman2019}. 
Even though being a relatively new language, initially released in 2012 for the broader public, \textit{Julia} gained much momentum over the last five decade, being used in various scientific disciplines CITE. \\

In the following, \textit{Julia} will be introduced as a language for scientifc computation. Section \ref{JM:sec:basics} starts by introducing how to define and analyse functions before introducing the concept of multiple dispatch to the reader.
Afterward, section \ref{JM:sec:JITOpt} explains benchmarking, optimizing code, and just-in-time compilation. Section \ref{JM:sec:GOMP} a simple greedy algorithm for sparse regression is implemented, showcasing the ease of use to transpile mathematical notation to fast code.
Section \ref{JM:sec:ECO} highlights some of the most useful and prominent packages of the language.
A conclusion is drawn in section \ref{JM:sec:CONC}.

\newpage
