\section{Basics and Multiple Dispatch}
\label{JM:sec:basics}

There exist different ways to define functions in \text{Julia}. Similar to e.g. \text{Python} or \text{Matlab}, a simple Gaussian radial basis function $f : \mathbb{R} \times \mathbb{R} \mapsto \mathbb{R}$ 
can be implemented as follows

\lstinputlisting[language=Julia,firstline=1, lastline=8]{../scripts/HelloWorld.jl}

As we can see, we defined no input types for the function and called it with a Float64 and Int64 as arguments, resulting in a Float64 as the best intersection of both number types.
Changing the input type of the first argument to Float32 returns Float32, again using the intersection of the input data types. To see how the evaluation works, we will use

\lstinputlisting[language=Julia,firstline=10, lastline=10]{../scripts/HelloWorld.jl}

Which will evaluate the function and returns a lowered and type-inferred abstract syntax tree for the method

\begin{lstlisting}[language=Julia]
    CodeInfo(
        1 ─ %1 = Base.sitofp(Float64, y)::Float64
        │   %2 = Base.sub_float(x, %1)::Float64
        │   %3 = Base.mul_float(%2, %2)::Float64
        │   %4 = Base.neg_float(%3)::Float64
        │   %5 = invoke Main.exp(%4::Float64)::Float64
        └──      return %5
        ) => Float64
\end{lstlisting}

The above output contains two important informations: the intermediate representation (IR), indicated by the lines with a leading \%, passed on to the compiler and the type information
propagated through the function. 
It first the value of $y$ is converted into a Float64 (\%1), the difference $x-y$ is computed (\%2), multiplied with itsself (\%3), negated (\%4) and mapped via the exponential function (\%5).
Entering different arguments also works out of the box. Using a pair of ComplexFloat64 and Float32 
returns a complex-valued result, as can be expected. 
Different representations as the one shown above can be accessed via {@code\_lowered}, @{code\_typed}, @{code\_lowered} and @{code\_native}. These return the IR, IR with type information, the code for the compiler and the machine code respectively.

\lstinputlisting[language=Julia,firstline=12, lastline=12]{../scripts/HelloWorld.jl}

The reason why this approach works is called multiple dispatch (MD) and a core design element of \textit{Julia}. Most prominent languages operate on
obect-oriented programming (OOP) or use functional programming\footnote{FP refers to purely functional programming in the scope of this work. In general modern languages using FP as a paradigm, like \textit{Closure}, tend to use MD.} (FP).
In FP, each function is defined uniquely by its name and stored in the global namespace. OOP stores methods related to objects in the namespace
of the corresponding object definition, allowing for more expressiveness and clarity in defining the functions.

Following \cite{JMKarpinski2019}, the expressiveness of a language can be defined as the number of different methods with the same name but different purposes depending on their arguments.
Consider the exponential function $exp : \mathbb{X} \mapsto \mathbb{X},~ \mathbb{X} \in \lbrace \mathbb{R}, \mathbb{C}, \dots \rbrace$. FP would define several methods by name for different 
arguments, e.g. $exp_{\mathbb{R}}$, $exp_{\mathbb{C}}$. In OOP, the object itself is the dispatch argument, so a pseudo-code notation would result in $\mathbb{R}.exp$, $\mathbb{C}.exp$.
In MD, all input arguments are checked, hence $exp$ can be used and extended for all types it holds a definition for. These insights are summarized in Table \ref{JM:tab:ExpressivePow}, showing the proportional
growth of expressiveness in relation to their dispatch arguments. \\

\begin{table}
    \begin{tabular}{|l|l|l|l|} \hline
        & FP & OOP & MD \\ \hline
        Dispatch arguments & $\emptyset$ & $\lbrace x_1 \rbrace$ & $\lbrace x_1 , x_2, \dots, x_N \rbrace$ \\
        Expressive Order &  $\mathcal{O}\left(1\right)$ & $\mathcal{O}\left(\left|x_1\right|\right)$ &  $\mathcal{O}\left(\prod_{i=1}^N \left|x_i\right|\right)$ \\
        Expressive Power & const. & linear & exponential \\ \hline
    \end{tabular}
    \caption{Expressiveness of different programming paradigms}
    \label{JM:tab:ExpressivePow}
\end{table}

If a general function, e.g. $+$, is called in \textit{Julia}, the compiler automatically uses the right function definition based on the arguments. 
This can be confirmed with


\begin{lstlisting}[language=Julia]
    methods(+, (Number,Number))
    # 40 methods for generic function "+":
    [1] +(x::T, y::T) where T<:Union{Int128, ..., UInt8} in Base at int.jl:87
    [2] +(c::Union{UInt16, UInt32, UInt64, UInt8}, x::BigInt) in Base.GMP at gmp.jl:528
    [3] +(c::Union{Int16, Int32, Int64, Int8}, x::BigInt) in Base.GMP at gmp.jl:534
    [4] +(c::Union{UInt16, UInt32, UInt64, UInt8}, x::BigFloat) in Base.MPFR at mpfr.jl:376
    [5] +(c::Union{Int16, Int32, Int64, Int8}, x::BigFloat) in Base.MPFR at mpfr.jl:384
    [6] +(c::Union{Float16, Float32, Float64}, x::BigFloat) in Base.MPFR at mpfr.jl:392
    ... 
\end{lstlisting}

which returns all methods defined in the current workspace for the $+$ operator using two Number types as inputs. Number represents an abstract type spanning various number types, e.g. Float, Int, UInt, Complex. 
Evaluating the following code block 

\lstinputlisting[language=Julia,firstline=14, lastline=16]{../scripts/HelloWorld.jl}

returns the corresponding method used for computation and where it is defined.

\begin{lstlisting}[language=Julia]
    exp(x::T) where T<:Float32 in Base.Math at special/exp.jl:229
    exp(z::Complex) in Base at complex.jl:638
\end{lstlisting}

A key difference to OOP based programming is that no method definition has to explicitly know of the other, as long as it is unique in its dispatch arguments\footnote{An unwanted dispatch on an already defined method can result unstable behaviour and called \textit{type-piracy}}.
This allows developers to easily extend existing packages and functionalities, allowing excessive reuse of the codebase\footnote{A good example is given by \url{https://github.com/JuliaPlots/RecipesBase.jl}, which is used to define individual recipes used for plotting without explicitly depending on \textit{Plots.jl}}. 
As a small example consider defining a new Number type and its corresponding $\exp$:

\lstinputlisting[language=Julia,firstline=19, lastline=27]{../scripts/HelloWorld.jl}

To make the function $f$ useable for vector operations, multiple dispatch can be leveraged as well:

\lstinputlisting[language=Julia,firstline=29, lastline=38]{../scripts/HelloWorld.jl}

Since both $\exp$ and \^~ are not defined for vectors, we use broadcasting to call the function on each element of the data structure individually. This is accomplished via the 
. operator. Checking the available definitions of the function via 

\lstinputlisting[language=Julia,firstline=40, lastline=40]{../scripts/HelloWorld.jl}

we can see that two definitions are present

\begin{lstlisting}[language=Julia]
    # 2 methods for generic function "f":
    [1] f(x::AbstractVector{T} where T, y::AbstractVector{T} where T) in Main at ./HelloWorld.jl:32
    [2] f(x, y) in Main at ./HelloWorld.jl:3
\end{lstlisting}

[1] is the definition we just added and [2] our original implementation from the beginning of this section, extending the function to be reused and avoiding different naming schemes.

\newpage




