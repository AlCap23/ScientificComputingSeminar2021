\section{Conclusion}
\label{JM:sec:CONC}

\textit{Julia} set out to offer high-performance computation while providing clean and easily readable syntax, tackling the \textit{two language problem} common 
in today's software. As this report shows, most of these promises have been fulfilled. Focusing on essential features for numerical linear algebra in the core installation, it readily provides necessary 
tools to conduct state-of-the-art research. MD allows for effective reuse and extension of methods and types, leading to a growing amount of 
packages that form a densely connected ecosystem with specialized scopes but familiar notation. 
Many of the algorithms provided reach high performance or are easily boosted via accessible code or type transformations via macros. While being fast and offering a versatile collection of algorithms on its own, \textit{Julia} offers wrapper packages from and to 
several languages. This enables easy reuse of other sources and generation of APIs. 
While it can be used as a general programming language, most of the recently published research is still focused on classical 
fields like numerics, simulation, and control or physics, see e.g. \cite{JMJarlebring2019, JMForetsEtAl2020, JMKalubaEtAl2021}.\\

An important, missing feature is the lack of support to generate independent executables and no hard real-time guarantees\footnote{To the best knowledge of the author.}. Exceptions, like \cite{JMKoolenDeits2019}, exist but are rare.
In addition, \textit{Julia} is still early in its maturation cycle, which still lowers its acceptance rate and userbase. \textit{Python} became prominent around 2010, nearly 20 years after its initial release. 
This might increase the difficulties of many new users who have been trained to code in an OOP paradigm, which can make moving to \textit{Julia} challenging.






